%%\documentclass[hyperref={pdfpagelabels=false},compress]{beamer}
\documentclass[10pt]{beamer}

\usetheme{default}
%\usetheme{metropolis}
\useoutertheme{noslidenum}
\usefonttheme[onlylarge]{serif}
% \usecolortheme{beaver}
\usecolortheme{UCLAsquirrel}
%\logo{\includegraphics[height=0.7cm]{/home/sudiptob/texmf/tex/latex/uclabeamer/logo_ucla_cw.png}}

\setbeamerfont*{frametitle}{size=\normalsize,series=\bfseries}
\setbeamertemplate{navigation symbols}{}

\usepackage{lmodern}
%\usepackage{subfigure}
\usepackage{subfig}
\usepackage{pifont}
\usepackage{tabu}
\usepackage{xcolor}
\usepackage{algorithm}
\usepackage{algpseudocode}
%\usepackage{enumitem}
\usepackage{remreset}
\usepackage{etoolbox}
\usepackage{comment} % end and begin comment
%\usepackage{dtklogos} 
\usepackage{listings}
\lstset{breaklines=true} 
\usepackage{cancel}
\usepackage{epstopdf}
\usepackage{overpic}
\usepackage{url}
\usepackage{calrsfs}
\usepackage{mathrsfs}
\usepackage{epsfig}
\usepackage{cancel}
\usepackage{changepage}

\usepackage[english]{babel}
\usepackage[latin1]{inputenc}
\usepackage{times}
\usepackage[T1]{fontenc}
\usepackage{tikz}
\usetikzlibrary{arrows}
\tikzstyle{block}=[draw opacity=0.7,line width=1.4cm]
\let\code=\texttt
\let\proglang=\textsf

\newcommand{\gline}{\textcolor{gray}{\hline}}
\newcommand{\cmark}{\ding{51}}%
\newcommand{\xmark}{\ding{55}}%
\newcommand{\gcheck}{\textcolor{blue}{\Large \cmark}}
\newcommand{\rcross}{\textcolor{red}{\Large \xmark}}
\newcommand{\tkt}{\tilde{K}_\theta}
\newcommand{\kt}{K_\theta}
\newcommand{\ind}{\overset{ind}{\sim}}
\newcommand{\plim}{\overset{p}{\rightarrow}}
\newcommand{\cx}{\frac {X'X}n}
\newcommand{\cz}{\frac {Z'Z}n}
\newcommand{\ccz}{\frac {Z'Z}n - \Sigma_A}
\newcommand{\czy}{\frac {Z'y}n}
\newcommand{\cyz}{\frac {y'Z}n}
\newcommand{\cxy}{\frac {X'y}n}
\newcommand{\cyx}{\frac {y'X}n}
\newcommand{\myitem}{\vskip3mm \item}
%\newcommand{\iid}{\overset{iid}{\sim}}

\newcommand{\calS}{{\cal S}}
\newcommand{\calA}{{\cal A}}
\newcommand{\calK}{{\cal K}}
\newcommand{\calX}{{\cal X}}
\newcommand{\calD}{{\cal D}}
\newcommand{\calG}{{\cal G}}
\newcommand{\calT}{{\cal T}}
\newcommand{\calU}{{\cal U}}
\newcommand{\calR}{{\cal R}}
\newcommand{\tp}{\tilde{p}}
\newcommand{\tildebC}{\tilde{\bC}}
\newcommand{\calL}{{\cal L}}
\newcommand{\calP}{{\cal P}}

\newcommand{\given}{\, | \,}


%\newcommand{\T}{\top}

\newcommand{\blue}[1]{{\color{RoyalBlue!90} #1}}
\newcommand{\red}[1]{{\color{Red} #1}}
\newcommand{\green}[1]{{\color{Green} #1}}
\newcommand{\titl}[1]{{\begin{large}\begin{center}#1\end{center}\end{large}}}

\title[BIOSTAT~241] 
{%
Spatial Modeling for Areal Data: Spatial Autoregression
}

\author[Sudipto Banerjee]
{
 Sudipto Banerjee
}

\institute[UCLA]
{
  University of California, Los Angeles, USA
}

\date[]{}

\begin{document}
\begin{frame}
 \titlepage
\end{frame}

\begin{frame}{Hierarchical GLM for Spatial disease mapping}
	\begin{itemize}
		\item At unit (region) $i$, we observe response $y_i$ and  covariate $x_i$
		\item $g(E(y_i)) = x^{\top}_i\beta + w_i $ where $g(\cdot)$ denotes a suitable link function
%		\metroset{block=fill}
		\begin{block}{}
		\[
		p_2(\beta,\tau_w, \rho) \times N(w \given 0, \tau_w (D-\rho A)) \times \prod_{i=1}^n p_1(y_i\given x_i^{\top}\beta+w_i)
		\]
		\end{block}
		\item $p_1$ denotes the density corresponding to the link $g(\cdot)$
	\end{itemize}
\end{frame}

\begin{frame}{CAR models}
	\begin{itemize}
		\item CAR model:
		\begin{equation*}
		w_i \given w_{-i} \sim N\left(\frac \rho{n_i}\sum_{j \given i \sim j} w_j, \tau_w n_i\right)
		\end{equation*}
		\item Apply Brook's Lemma to obtain joint density for $w$
		\myitem $w=(w_1,w_2,\ldots,w_k)^{\top} \sim N(0, \tau_w (D-\rho A))$ where $D=\mbox{diag}(n_1,n_2,\ldots,n_k)$
		\only<1-2>{\myitem $\rho = 1 \Rightarrow$ Improper distribution as $(D-A)1 = 0$ \alert{(ICAR)}
		\begin{itemize}
		\item Can be still used as a prior for random effects
		\item Cannot be used directly as a data generating model
		\end{itemize}}
		\only<2>{\myitem $\rho <1 \Rightarrow$ Proper distribution with added parameter flexibility}		
		%\myitem Often oversmooth spatial random effects
	\end{itemize}
\end{frame}

\begin{frame}{Multivariate Disease Mapping}
 
\begin{itemize}\setlength{\itemsep}{0.5cm}

\item $y_{ij}$ is the disease count for disease $j$ in county $i$.

\item $x_{ij}$ are explanatory, region-level spatial covariates for disease $j$.

%\item $E_{ij}$'s are expected number of cases (under homogeneity) for disease $j$ in county $i$

%\item For counts of (rare) diseases, poisson regression model:

% \begin{exampleblock}{}
% \[
% Y_{ij} \: \: \stackrel{ind} \sim \: \:
% Poisson(E_{ij}e^{x_{ij}^{\top}\beta_j + \: \phi_{ij}}) \; , \; \: i=1, \ldots, n, \: j=1, \ldots, p.
% \]
% \end{exampleblock}
\begin{block}{}
\[
p_2(\beta,\theta) \times N(\phi \given 0, \Sigma_{\theta}) \times \prod_{i=1}^n\prod_{j=1}^p p_1(y_{ij}\given x_{ij}^{\top}\beta_j + \phi_{ij})
\]
\end{block}

\item How do we model the $\phi_{ij}$'s?

\end{itemize}

\end{frame}


\begin{frame}{Multivariate disease mapping using graphs}
\begin{picture}(50, 100)         %% Slides use just picture environment
\put(60,50){\circle{50}} \put(160,50){\circle{50}}
\put(260,50){\circle{50}} \put(50,50){$\phi_{11}$}
\put(60,-50){\circle{50}} \put(160,-50){\circle{50}}
\put(260,-50){\circle{50}} \put(60,25){\line(0, -50){50}}
\put(160,25){\line(0, -50){50}}\put(260,25){\line(0, -50){50}}

\put(80,30){\line(5, -6){50}} \put(180,30){\line(5, -6){50}}
\put(240,30){\line(-6, -5){60}} \put(140,25){\line(-6, -5){60}}

\put(90,50){\line(20, 0){40}} \put(190,50){\line(20, 0){40}}
\put(90,-50){\line(20, 0){40}} \put(190,-50){\line(20, 0){40}}

\put(100,60){\small $\alpha_1$}\put(200,60){\small $\alpha_1$}
\put(100,-40){\small $\alpha_2$}\put(200,-40){\small $\alpha_2$}
\put(40,0){\small $\alpha_0$}\put(140,0){\small
$\alpha_0$}\put(270,0){\small $\alpha_0$}

\put(80,0){\small $\alpha_3$}\put(180,0){\small $\alpha_3$}
\put(110,20){\small $\alpha_3$}\put(210,20){\small $\alpha_3$}


\put(150,50){$\phi_{21}$}\put(250,50){$\phi_{31}$}
\put(50,-50){$\phi_{12}$}\put(150,-50){$\phi_{22}$}
\put(250,-50){$\phi_{32}$}

\put(30, 90){\small county 1}\put(130, 90){\small county
2}\put(230, 90){\small county 3}

\put(300, 50){\small A}\put(300, -50){\small B}
\end{picture}
\end{frame}

\begin{frame}{Conditional bivariate CAR: Jin et al. (2005)}
 \begin{itemize}
  \item Build a hierarchical bivariate spatial model for $p=2$ outcomes:
  \begin{align*}
   N(\phi_1 \given 0, \tau_1(D-\rho_1 A)) \times N(\phi_2 \given C\phi_1, \tau_2(D - \rho_2 A))
  \end{align*}

  \item $\mbox{E}[\phi_2\given \phi_1] = C\phi_1$. Assume that the elements of $C$ are 
\[
c_{ij}=\left\{\begin{array}{ll}   \eta_0 & \mbox{if $j=i$} \\
\eta_{1} &  \mbox{if $j \sim i$ (i.e., if region $j$ is a neighbor of region $i$)} \\
0 & \mbox{otherwise}
\end{array} \right. \; .
\]

 \item $C = \eta_0 I + \eta_1 A$, where $\eta_0$ and $\eta_1$ control spatial smoothing for cross-covariances.
 
 \item Call this $\mbox{CBCAR}(\rho_1, \rho_2, \eta_0, \eta_1, \tau_1, \tau_2)$. Some special cases emerge:
    \begin{itemize}
     \item Separable MCAR     
     \item Kim, Sun and Tsutakawa (2001): Two-fold CAR model with smoothing of cross-correlations
    \end{itemize}
 \item Generalizations: $C = \sum_{j} \eta_j A^{j}$   
 \end{itemize}

\end{frame}

\end{document}
