%\documentclass[hyperref={pdfpagelabels=false},compress]{beamer}

% \titlegraphic{\hfill\includegraphics[height=1.5cm]{logo.pdf}}

\documentclass[xcolor=pdftex,dvipsnames,table,numbers,hyperref={pdfpagelabels=false},compress]{beamer}

\usepackage{amsmath}
\usepackage{graphicx}
\usepackage{amsfonts}
\usepackage{amssymb}
\usepackage{amssymb}
\usepackage{tabularx}
\usepackage{epstopdf}
\usepackage{overpic}
\usepackage{url}
\usepackage{calrsfs}
\usepackage{mathrsfs}
\usepackage{epsfig}
\usepackage{cancel}
\usepackage{changepage}

\usepackage{lmodern}
%\usepackage{mystyle}
\usepackage{subfig}
%\usepackage{subfigure}
\usepackage{pifont}
\usepackage{tabu}
\usepackage{xcolor}
\usepackage{algorithm}
\usepackage{algpseudocode}
%\usepackage{enumitem}
\usepackage{remreset}
\usepackage{etoolbox}
\usepackage{comment} % end and begin comment
%\usepackage{dtklogos} 
\usepackage{listings}
\lstset{breaklines=true} 

\newcommand{\gline}{\textcolor{gray}{\hline}}
\newcommand{\cmark}{\ding{51}}%
\newcommand{\xmark}{\ding{55}}%
\newcommand{\gcheck}{\textcolor{blue}{\Large \cmark}}
\newcommand{\rcross}{\textcolor{red}{\Large \xmark}}
\newcommand{\tkt}{\tilde{K}_\theta}
\newcommand{\kt}{K_\theta}
\newcommand{\ind}{\overset{ind}{\sim}}
\newcommand{\plim}{\overset{p}{\rightarrow}}
\newcommand{\cx}{\frac {X'X}n}
\newcommand{\cz}{\frac {Z'Z}n}
\newcommand{\ccz}{\frac {Z'Z}n - \Sigma_A}
\newcommand{\czy}{\frac {Z'y}n}
\newcommand{\cyz}{\frac {y'Z}n}
\newcommand{\cxy}{\frac {X'y}n}
\newcommand{\cyx}{\frac {y'X}n}
\newcommand{\myitem}{\vskip3mm \item}

\newcommand{\calS}{{\cal S}}
\newcommand{\calA}{{\cal A}}
\newcommand{\calK}{{\cal K}}
\newcommand{\calX}{{\cal X}}
\newcommand{\calD}{{\cal D}}
\newcommand{\calG}{{\cal G}}
\newcommand{\calT}{{\cal T}}
\newcommand{\calU}{{\cal U}}
\newcommand{\calR}{{\cal R}}
\newcommand{\tp}{\tilde{p}}
\newcommand{\tildebC}{\tilde{\bC}}
\newcommand{\calL}{{\cal L}}

\newcommand{\blam}{ \mbox{\boldmath $ \lambda $} }
\newcommand{\bet}{ \mbox{\boldmath $ \eta $} }
\newcommand{\bome}{ \mbox{\boldmath $ \omega $} }
\newcommand{\bbet}{ \mbox{\boldmath $ \beta $} }
\newcommand{\bbeta}{ \mbox{\boldmath $ \beta $} }
\newcommand{\balph}{ \mbox{\boldmath $ \alpha $} }
\newcommand{\balpha}{ \mbox{\boldmath $ \alpha $} }
\newcommand{\bphi}{ \mbox{\boldmath $\phi$}}
\newcommand{\bzeta}{ \mbox{\boldmath $\zeta$}}
\newcommand{\bkap}{ \mbox{\boldmath $\kappa$}}
\newcommand{\bkappa}{ \mbox{\boldmath $\kappa$}}
\newcommand{\beps}{ \mbox{\boldmath $\epsilon$}}
\newcommand{\bepsilon}{ \mbox{\boldmath $\epsilon$}}
\newcommand{\bthet}{ \mbox{\boldmath $ \theta $} }
\newcommand{\btheta}{ \mbox{\boldmath $ \theta $} }
\newcommand{\blambda}{ \mbox{\boldmath $ \lambda $} }
\newcommand{\bnu}{ \mbox{\boldmath $\nu$} }
\newcommand{\bmu}{ \mbox{\boldmath $\mu$} }
\newcommand{\bGam}{ \mbox{\boldmath $\Gamma$} }
\newcommand{\bSig}{ \mbox{\boldmath $\Sigma$} }
\newcommand{\bSigma}{ \mbox{\boldmath $\Sigma$} }
\newcommand{\bPhi}{ \mbox{\boldmath $\Phi$} }
\newcommand{\bThet}{ \mbox{\boldmath $\Theta$} }
\newcommand{\bTheta}{ \mbox{\boldmath $\Theta$} }
\newcommand{\bDel}{ \mbox{\boldmath $\Delta$} }
\newcommand{\bDelta}{ \mbox{\boldmath $\Delta$} }
\newcommand{\bnabla}{ \mbox{\boldmath $\nabla$} }
\newcommand{\bLam}{ \mbox{\boldmath $\Lambda$} }
\newcommand{\bLambda}{ \mbox{\boldmath $\Lambda$} }
\newcommand{\bgam}{ \mbox{\boldmath $\gamma$} }
\newcommand{\bgamma}{ \mbox{\boldmath $\gamma$} }
\newcommand{\brho}{ \mbox{\boldmath $\rho$} }
\newcommand{\bdel}{ \mbox{\boldmath $\delta$} }
\newcommand{\bdelta}{ \mbox{\boldmath $\delta$} }
\newcommand{\sis}{\sigma^2}
\newcommand{\bOmega}{\mbox{\boldmath $\Omega$} }
\newcommand{\bPsi}{ {\boldsymbol \Psi} }

\newcommand{\bzero}{\textbf{0}}
\newcommand{\bones}{\textbf{1}}
\newcommand{\ba}{\textbf{a}}
\newcommand{\bb}{\textbf{b}}
\newcommand{\bB}{\textbf{B}}
%\newcommand{\bA}{\textbf{A}}
\newcommand{\bc}{\textbf{c}}
\newcommand{\bC}{\textbf{C}}
\newcommand{\bA}{\textbf{A}}
\newcommand{\bd}{\textbf{d}}
\newcommand{\bD}{\textbf{D}}
\newcommand{\be}{\textbf{e}}
\newcommand{\bE}{\textbf{E}}
\newcommand{\bk}{\textbf{k}}
\newcommand{\bK}{\textbf{K}}
\newcommand{\bh}{\textbf{h}}
\newcommand{\bs}{\textbf{s}}
\newcommand{\bS}{\textbf{S}}
\newcommand{\bH}{\textbf{H}}
\newcommand{\bI}{\textbf{I}}
\newcommand{\bt}{\textbf{t}}
\newcommand{\bu}{\textbf{u}}
\newcommand{\bv}{\textbf{v}}
\newcommand{\bw}{\textbf{w}}
\newcommand{\bW}{\textbf{W}}
\newcommand{\bx}{\textbf{x}}
\newcommand{\bX}{\textbf{X}}
\newcommand{\by}{\textbf{y}}
\newcommand{\bY}{\textbf{Y}}
\newcommand{\bz}{\textbf{z}}
\newcommand{\bZ}{\textbf{Z}}
\newcommand{\bL}{\textbf{L}}
\newcommand{\br}{\textbf{r}}
\newcommand{\bR}{\textbf{R}}

\newcommand{\given}{\,|\,}
\newcommand{\T}{\top}

\newcommand{\blue}[1]{{\color{RoyalBlue!90} #1}}
\newcommand{\red}[1]{{\color{Red} #1}}
\newcommand{\green}[1]{{\color{Green} #1}}
\newcommand{\titl}[1]{{\begin{large}\begin{center}#1\end{center}\end{large}}}

\newcommand{\tildea}{\tilde{a}}
\newcommand{\tildeba}{\tilde{\ba}}
\newcommand{\tildebv}{\tilde{\bv}}
\newcommand{\tildev}{\tilde{v}}
\newcommand{\tildeA}{\tilde{A}}
\newcommand{\tildeC}{\tilde{C}}
\newcommand{\tildeK}{\tilde{K}}
\newcommand{\tildew}{\tilde{w}}
\newcommand{\tildeu}{\tilde{u}}
\newcommand{\tildebw}{\tilde{\bw}}
\newcommand{\tildeeps}{\tilde{\epsilon}}
\newcommand{\tildebeps}{\tilde{\bepsilon}}
\newcommand{\eps}{\epsilon}
\newcommand{\sigs}{\sigma^2}
\newcommand{\taus}{\tau^2}
\newcommand{\iid}{\stackrel{\mathrm{iid}}{\sim}}

%\newcommand{\calS}{{\cal S}}
\newcommand{\calC}{{\cal C}}

%\documentclass[10pt]{beamer}

\usetheme{metropolis}
\usepackage{appendixnumberbeamer}

\usepackage{booktabs}
\usepackage[scale=2]{ccicons}

\usepackage{pgfplots}
\usepgfplotslibrary{dateplot}

\usepackage{xspace}
\newcommand{\themename}{\textbf{\textsc{metropolis}}\xspace}

\makeatletter
\@addtoreset{subfigure}{framenumber}% subfigure counter resets every frame
\makeatother

\makeatletter
\@addtoreset{figure}{framenumber}% subfigure counter resets every frame
\makeatother

\setbeamertemplate{caption}{\raggedright\insertcaption\par}
\captionsetup[subfigure]{labelformat=empty}


\title[]{Bayesian Geostatistics}
\author{Sudipto Banerjee}
	
\institute{
\begin{tiny}Department of  Biostatistics, Fielding School of Public Health, University of California, Los Angeles.\end{tiny}
}

\date{Spring, 2021}


\begin{document}

\maketitle


\begin{frame}{Bayesian spatial random effect models}
 
 \begin{itemize}
  \item Continuous data:
   \begin{multline*}
  y(s_i) \given \mu(s_i),\: \tau^2 \stackrel{ind}{\sim} N(\mu(s_i), \tau^2)\;;\quad i=1,2,\ldots,n\;; \\
  \mu(s_i) = \beta_0 + \beta_1 x_{1}(s_i) + \beta_2 x_{2}(s_i) + \cdots + \beta_p x_{p}(s_i) + w(s_i)\;; \\
  \beta_j \stackrel{ind}{\sim} N(0, \sigma^2_{\beta})\;;\quad j=0,1,\ldots,p\;; \\
  w = (w(s_1), w(s_2),\ldots, w(s_n))^{\top} \sim N(0,\sigma^2 R_w(\phi))\;;\\
  1/\tau^2 \sim \mbox{Gamma}(a_{\tau}, b_{\tau})\;;\quad
  1/\sigma^2 \sim \mbox{Gamma}(a_{\sigma},b_{\sigma})\;;\\
  \phi\sim \mbox{Unif}(a_{\phi}, b_{\phi})\;.
 \end{multline*}
\item $R_w(\phi)$ is $n\times n$ spatial correlation matrix. 
 \end{itemize}

\end{frame}

\begin{frame}{Bayesian spatial random effect models}
 
\begin{itemize}\setlength{\itemsep}{0.4cm}
 \item Count data:
 \begin{multline*}
   y(s_i) \sim Poi(\lambda(s_i))\;;\quad i=1,2,\ldots,n\;;\\
   \log\lambda(s_i) = \beta_0 + \beta_1 x_{1}(s_i) + \beta_2 x_{2}(s_i) + \cdots + \beta_p x_{p}(s_i) + w(s_i)\;; \\
   \beta_j \stackrel{ind}{\sim} N(0, \sigma^2_{\beta})\;; \quad j=0,1,\ldots,p\;;\quad w \sim N(0,\sigma^2R_w)\;; \\
   1/\sigma^2 \sim \mbox{Gamma}(a_{\sigma},b_{\sigma})\;;\quad \phi\sim \mbox{Unif}(a_{\phi}, b_{\phi})\;.
  \end{multline*}
 \item $R_w(\phi)$ is $n\times n$ spatial correlation matrix.   
\end{itemize}

\end{frame}

\begin{frame}{Bayesian spatial random effect models}
 
\begin{itemize}\setlength{\itemsep}{0.4cm}
 \item Binary data:
 \begin{multline*}
   y(s_i) \sim Ber(p(s_i))\;;\quad i=1,2,\ldots,n\;;\\
   \log\left(\frac{p(s_i)}{1-p(s_i)}\right) = \beta_0 + \beta_1 x_{1}(s_i) + \beta_2 x_{2}(s_i) + \cdots + \beta_p x_{p}(s_i) + w(s_i)\;; \\
   \beta_j \stackrel{ind}{\sim} N(0, \sigma^2_{\beta})\;; \quad j=0,1,\ldots,p\;;\quad w \sim N(0,\sigma^2R_w)\;; \\
   1/\sigma^2 \sim \mbox{Gamma}(a_{\sigma},b_{\sigma})\;;\quad \phi\sim \mbox{Unif}(a_{\phi}, b_{\phi})\;.
  \end{multline*}
 \item $R_w(\phi)$ is $n\times n$ spatial correlation matrix.   
\end{itemize}

\end{frame}

\begin{frame}{Spatial random effects: Gaussian process}
 
 \begin{itemize}
  \item We say that $w(s) \sim GP(0, \sigma^2\rho(\cdot))$:
  \[
   w = (w(s_1), w(s_2),\ldots, w(s_n))^{\top} \sim N(0,\sigma^2 R_w)\;;
  \]
  
  \item $R_w$ is $n\times n$ spatial correlation matrix:
  \[
   R_{w}[i,j] = \rho(s_i, s_j)\;.
  \]

  \item The correlation function is parametrized to capture strength of association as a function of distance. Practical choice (works well for a variety of situations):
  \[
   \rho(s_i, s_j) = \exp(-\phi \|s_i-s_j\|)\;.
  \]

 \end{itemize}
 
\end{frame}

\begin{frame}{Bayesian inference for continuous spatial data}
 
 \begin{itemize}
  \item Step-I: Estimate parameters (MCMC) by sampling from
  \[
   [\beta, w, \tau^2, \sigma^2, \phi \given y, X]
  \]

  \item Step-II: Estimate the latent process $w(s_0)$ at new location $s_0$ by sampling from
  \[
   [w(s_0) \given w, \sigma^2, \phi] 
  \]
  for each sampled value of $w$, $\sigma^2$ and $\phi$ obtained in Step-I.
  
  \item Step III: Obtain posterior samples of $\mu(s_0)$:
  \[
   \mu(s_0) = \beta_0 + \beta_1 x_{1}(s_0) + \beta_2 x_{2}(s_0) + \cdots + \beta_p x_{p}(s_0) + w(s_0)\;.
  \]

  \item Step-IV: Predict $y(s_0)$ by drawing its value from
  \[
   N(\mu(s_0),\tau^2)
  \]
  for each sampled $\mu(s_0)$ (from Step-III) and $\tau^2$ (from Step-I).
  
 \end{itemize}
 
\end{frame}

\begin{frame}{Bayesian inference for spatial count data}
 
 \begin{itemize}
  \item Step-I: Estimate parameters (MCMC) by sampling from
  \[
   [\beta, w, \sigma^2, \phi \given y, X]
  \]

  \item Step-II: Estimate the latent process $w(s_0)$ at new location $s_0$ by drawing from
  \[
   [w(s_0) \given w, \sigma^2, \phi] 
  \]
  for each sampled value of $w$, $\sigma^2$ and $\phi$ obtained in Step-I.
  
  \item Step III: Obtain posterior samples of $\lambda(s_0)$:
  {\small
  \[
   \lambda(s_0) = \exp\left(\beta_0 + \beta_1 x_{1}(s_0) + \beta_2 x_{2}(s_0) + \cdots + \beta_p x_{p}(s_0) + w(s_0)\right)\;.
  \]
  }
  \item Step-IV: Predict $y(s_0)$ by drawing its value from
  \[
   Poi(\lambda(s_0))
  \]
  for each sampled $\lambda(s_0)$ in Step-III.
  
 \end{itemize}
 
\end{frame}

\begin{frame}{Bayesian inference for binary count data}
 
 \begin{itemize}
  \item Step-I: Estimate parameters (MCMC) by sampling from
  \[
   [\beta, w, \sigma^2, \phi \given y, X]
  \]

  \item Step-II: Estimate the latent process $w(s_0)$ at new location $s_0$ by drawing from
  \[
   [w(s_0) \given w, \sigma^2, \phi] 
  \]
  for each sampled value of $w$, $\sigma^2$ and $\phi$ obtained in Step-I.
  
  \item Step III: Obtain posterior samples of $p(s_0)$:
  {\small
  \[
   p(s_0) = \mbox{logit}^{-1}\left(\beta_0 + \beta_1 x_{1}(s_0) + \beta_2 x_{2}(s_0) + \cdots + \beta_p x_{p}(s_0) + w(s_0)\right)\;.
  \]
}
  \item Step-IV: Predict $y(s_0)$ by drawing its value from
  \[
   Ber(p(s_0))
  \]
  for each sampled $p(s_0)$ in Step-III.
  
 \end{itemize}
 
\end{frame}

\begin{frame}{Setting Priors}

{\small
\begin{itemize}
 \item For regression slopes we customarily assign non-informative priors.
 
 \item For the variance component $\sigma^2$ (partial sill), we customarily choose an Inverse-Gamma (or, equivalently, Gamma prior for $1/\sigma^2$)---the shape parameter is taken to be $2$ and the scale parameter is chosen so that the prior mean is equal to the scale parameter. This value can be set from an exploratory variogram analysis. Strategy for $\tau^2$ (nugget) is similar.
 
 \item For the range parameter $\phi$, we usually set it so that the effective range (distance where spatial correlation drops to 0.05) is between some small number and does not exceed about 50\% of the maximum inter-site distance. For example, with the exponential correlation function we solve $\rho(d;\phi) = 0.05$ and see that $\phi \approx 3/d$, where $d$ is the effective spatial range. We bound $d \in (d_{\min}, d_{\max})$ and this suggests $\phi \sim \mbox{Unif}(3/d_{\max}, 3/d_{\max})$. 
\end{itemize}
}

\end{frame}

\end{document}
